%%% ============================================================================================================= %%%
%%%                                  The lastest updated date: March, 2009                                    %%%
%%% ============================================================================================================= %%%
%%% Update Log: 11/16/2004, add the \RomanNumber and \romannumber for capital and small Roman numbers
%%%                         add \rb for table raising box
%%% Update Log: 05/08/2004, add the matlab and simulink definition in the end for reference
%%% Update Log: 04/26/2004, add the math operator ``dist'', update the definition for theorem-like enviroments
%%% Update Log: 03/04/2004, use package 'upgreek' for constants, e.g., $\uppi$ = 3.1415926.... (ISO standard)
%%% Update Log: 09/12/2004, update for solving the conflicts between amsthm and cssconf
%%% Update Log: 05/03/2009, merging with Luis' abbreviation setting
%%% Update Log: 01/20/2010, adding H option for floating figures
%%% ============================================================================================================= %%%

\usepackage{latexsym, amssymb, amsmath}
\usepackage[dvips]{graphicx} % added for inserting the pictures

%%%%%%%%%%%%%%%%
\usepackage{float}      %these two packages are for fixing figures in place
\restylefloat{figure}
%%%%%%%%%%%%%%%%

\usepackage{graphicx,dsfont}
%\usepackage{eufrak} % for conjugate exponents
\usepackage{mathrsfs}  % package for math fonts: mathscripts \mathscr
\usepackage{color} % standard color package
%\usepackage{natbib}
\usepackage[latin1]{inputenc}
%% \usepackage[dvips]{color}
%% \usepackage{amsthm}
%% \usepackage{amscd}% amscd for "commutative diagram"
\usepackage{enumerate}
%% \usepackage{url}
%\usepackage{upref} % make all ref's are upright fonttype, no use in this paper % zzz: removed by Dr. Gray and Dr. Gonzalez
%% \usepackage{upgreek} % use upright greek to represent constant values, i.e., $\uppi$ = 3.14, ...
\usepackage{flushend} % align at the last lines of every page
%% \usepackage{easybmat} % A simple package for writing block matrices with equal column widths or equal rows heights or both, with various kinds of rules between rows and columns.
%% \usepackage{txfonts}%, pxfonts
%\usepackage{euscript} %, amsfonts}
%% \usepackage{times}
\usepackage{stfloats}
%% \usepackage{float}
%% Packages added by Dr. Gonzalez
%\usepackage{indentfirst}    %Automatically indents the beginning of the paragraph following a Latex section
%\usepackage{cite}%package to reduce the length of long citations by usinghyphen

%% Added by Luis
\usepackage{balance}
%%% ============================================================================================================= %%%

% Hong's setting for figures for this paper
%\setlength{\textfloatsep}{.3em}
%\setlength{\intextsep}{.5em}

% Reduce Space Between Paragraphs
%\setlength{\parskip}{1.5ex} %plus 1pt minus 1pt}

% Reduce Space Above and Below Equations
%\setlength{\jot}{-0.7pt} % eqnarray extra space
%\setlength{\abovedisplayskip}{-1pt}% plus 1pt minus 1pt} % extra space above eqn
%\setlength{\abovedisplayshortskip}{-1pt}% plus 1pt minus 1pt} % extra space above eqn
%\setlength{\belowdisplayskip}{-1pt}% plus 1pt minus 1pt} % extra space below eqn
%\setlength{\belowdisplayshortskip}{-1pt}% plus 1pt minus 1pt} % extra space below eqn
%\setlength{\topsep}{0.2pt}

% Insert Breaks Between a Long Formula
\allowdisplaybreaks[4] % added by Hong for allowing page breaks among a large formula


%%% ============================================================================================================= %%%
% new theorems environment
%\swapnumbers
%% \theoremstyle{plain} % default
\newtheorem{prop}{Proposition} % [section]
\newtheorem{proposition}[prop]{Proposition} % [section]
\newtheorem{lem}{Lemma}%[section]
\newtheorem{lemma}[lem]{Lemma} % [section]
\newtheorem{thm}{Theorem} % [section]
\newtheorem{theorem}[thm]{Theorem} % [section]
\newtheorem{cor}{Corollary} % [section]
\newtheorem{corollary}[cor]{Corollary} % [section]
%\newtheorem{exer}{Exercise}[section]
%\newtheorem{test}{Test}[section]

%% \theoremstyle{definition}
\newtheorem{defn}{Definition} % [section]
\newtheorem{defi}[defn]{Definition} % [section]
\newtheorem{definition}[defn]{Definition} % [section]
\newtheorem{conj}{Conjecture} % [section]
\newtheorem{exmp}{Example} % [section]
\newtheorem{example}[exmp]{Example} % [section]
\newtheorem{exam}[exmp]{Example} % [section]
\newtheorem{exam*}{Example}





\def\custombibliography#1{
 \normalsize
% The part was commented by Hong
% \begin{center}
% {\Large \bf{References}}
% \end{center}
\section*{\centering References}
 \list
 {[\arabic{enumi}]}{\settowidth\labelwidth{[#1]}\leftmargin\labelwidth
 \setlength{\itemsep}{.1em}
 \advance\leftmargin\labelsep
 \usecounter{enumi}}
 \def\newblock{\hskip .11em plus .33em minus -.07em}
 \sloppy
 \sfcode`\.=1000\relax}
\let\endthebibliography=\endlist

%%% ============================================================================================================= %%%
% Gray's Abbreviations

\font\grg=eurm10 \def\umu{{\hbox{\grg\char22}}} % upright mu.
\font\grs=eurm10 at 9pt \def\smu{{\hbox{\grs\char22}}} % use for micro...
\def\L2{{\cal L}_2}
\def\bull{\rule{0.08in}{0.08in}} % square filled bullet
\def\openbull{\framebox[0.08in][c]{$\;$}} % square unfilled bullet
%\def\re{{\rm I\! R}} % real numbers
%\def\nat{{\mathbb N}} % natural numbers (AMS symbol)
\def\re{{\mathbb R}} % real numbers (AMS symbol)
\def\card{{\rm card}}
\def\C{{\mathbb C}} % complex numbers (AMS symbol)
\def\Cc{{\cal C}} % class of continuous functions
\def\qed{\hfill$\Box \Box \Box$}
\def\shuffle{{\scriptscriptstyle \;\sqcup \hspace*{-0.05cm}\sqcup\;}}
\def\shuffleXW{{{\shuffle}}}
\def\allpoly{\mbox{$\re\langle X \rangle$}}
\def\allseries{\mbox{$\re \langle\langle X \rangle\rangle$}}
\def\allseriesLC{\mbox{$\re_{LC}\langle\langle X \rangle\rangle$}}
\def\allseriesm{\mbox{$\re^m\langle\langle X \rangle\rangle$}}
\def\allseriesmLC{\mbox{$\re^{m}_{LC} \langle\langle X \rangle\rangle$}}
\def\allseriesell{\mbox{$\re^{\ell} \langle\langle X \rangle\rangle $}}
\def\allseriesXWell{\mbox{$\re^{\ell} \langle\langle XW \rangle\rangle$}}
\def\allseriesellLC{\mbox{$\re^{\ell}_{LC}  \langle\langle X \rangle\rangle$}}
\def\allseriesXW{\mbox{$\re  \langle\langle XW \rangle\rangle $}}
\def\allseriesXWLC{\mbox{$\re_{LC} \langle\langle XW \rangle\rangle $}}
\def\allseriesWLC{\mbox{$\re_{LC}\langle\langle W \rangle\rangle$}}
\def\allseriesXWellLC{\mbox{$\re^{\ell}_{LC}\langle\langle XW \rangle\rangle$}}
\def\allseriesI{\mbox{$\re\langle\langle I \rangle\rangle$}}
\def\allseriesILC{\mbox{$\re_{LC} \langle\langle I \rangle\rangle$}}
\def\allseriesIm{\mbox{$\re^m  \langle\langle I \rangle\rangle$}}
\def\allseriesImLC{\mbox{$\re^m_{LC}\langle\langle I \rangle\rangle$}}
\def\allseriesIell{\mbox{$\re^{\ell}\langle\langle I \rangle\rangle$}}
\def\allseriesIellLC{\mbox{$\re^{\ell}_{LC} \langle\langle I \rangle\rangle$}}
\def\allseriesXOm{\mbox{$\re^m  \langle\langle X_0 \rangle\rangle$}} % use oh since zero is not allowed
\def\allseriesXOmLC{\mbox{$\re^m_{LC} \langle\langle X_0\rangle\rangle$}}
\def\allseriesXOellLC{\mbox{$\re^{\ell}_{LC}\langle\langle X_0 \rangle\rangle$}}
\def\allpoly{\mbox{$\re\langle X \rangle$}}
\def\allpolyW{\mbox{$\re \langle W \rangle$}}
\def\allpolyXW{\mbox{$\re\langle XW \rangle$}}
\def\mbf#1{\hbox{\mathversion{bold}$#1$}} % math boldface
\def\bfem#1{{\bf \em #1}} % boldface italics
\def\ns#1{{\textstyle #1}} % normalsize fonts in subscripts
\def\ss#1{{\scriptstyle #1}} % fonts in subsubscripts
\def\sss#1{{\scriptscriptstyle #1}} % fonts in subsubsubscripts
\def\eqref#1{(\ref{#1})} % parentheses around referenced equation numbers
\def\spacebox#1{\raisebox{-3pt}[4pt][8pt]{#1}} % puts extra space around items in a table
\def\spaceboxbelow#1{\raisebox{0pt}[1pt][7pt]{#1}} % puts extra space around items in a table
\def\ve{\varepsilon}
\def\sameau{\rule[0.017in]{0.2in}{0.012in}}
\def\innprod#1#2{\left\langle #1, #2 \right\rangle}
\def\abs#1{\left\vert #1 \right\vert}
\def\norm#1{\left\Vert#1\right\Vert}
\def\ord{{\rm ord}}
\def\supp{{\rm supp}}
\def\Tr{{\rm Tr}}
\def\vec{{\rm vec}}
\def\dist{{\rm dist}}
\def\fac{{\rm fac}}
\def\doubleone{{\rm\, l\!l}}
\def\modcomp{\:\tilde{\circ}\,} % modified composition product
\def\mbf#1{\hbox{\mathversion{bold}$#1$}} % math boldface
\newcommand{\comment}[1]{} % Allows one to comment out a block of text
\newcommand{\markred}[1]{\textcolor[rgb]{0.98,0.00,0.00}{#1}}
\newcommand{\commentforyuan}[1]{\textcolor[rgb]{0.00,0.00,1.00}{#1}}


%%% ============================================================================================================= %%%
% Gray's Environment Abbreviations
\def\begce{\begin{center}}
\def\endce{\end{center}}
\def\begar{\begin{array}}
\def\endar{\end{array}}
\def\begeq{\begin{equation}}
\def\endeq{\end{equation}}
\def\begdi{\begin{displaymath}}
\def\enddi{\end{displaymath}}
\def\begdis{\begin{eqnarray*}}
\def\enddis{\end{eqnarray*}}
\def\begeqa{\begin{eqnarray}}
\def\endeqa{\end{eqnarray}}
\def\begdes{\begin{description}}
\def\enddes{\end{description}}
\def\begit{\begin{itemize}}
\def\endit{\end{itemize}}
\def\begen{\begin{enumerate}}
\def\enden{\end{enumerate}}
\def\beglar{\left[\begin{array}}
\def\endrar{\end{array}\right]}
\def\begle{\begin{lemma}}
\def\endle{\end{lemma}}
\def\begde{\begin{definition}}
\def\endde{\end{definition}}
\def\begth{\begin{theorem}}
\def\endth{\end{theorem}}
\def\begco{\begin{corollary}}
\def\endco{\end{corollary}}
\def\begprop{\begin{proposition}}
\def\endprop{\end{proposition}}
\def\begex{\begin{example}}
\def\endex{\hfill\openbull \end{example} \vspace*{0.1in}}
%\def\endex{\begin{flushright} \hfill\openbull \end{flushright} \end{example} \vspace*{0.1in}}
\def\begexer{\begin{exercise}}
\def\endexer{\end{exercise}}
\def\begre{\noindent{\em Remark}: }
\def\endre{\\}
\def\begres{\noindent{\bf Remarks}:\begin{enumerate}}
\def\endres{\end{enumerate} \par}
\def\begpr{\noindent{\em Proof:}$\;\;$}
\def\endpr{\hfill\bull \vspace*{0.1in}}
\def\begproposi{\noindent{\em Proof of proposition}$\;\;$}
\def\endproposi{\hfill\bull \vspace*{0.1in}}
\def\begtab{\begin{tabular}}
\def\endtab{\end{tabular}}
\def\rref#1{(\ref{#1})}
\def\problem#1{\vspace*{0.1in}\noindent {\bf Problem \ref{#1}}} % for the solution key
%\def\begpr{\noindent{\em Proof}\ : } % changed by Hong
%\def\endpr{\hspace*{0.05in}\bull\vspace*{0.15in}\\}
%\def\beglar{\left[\begin{array}}
%\def\endrar{\end{array}\right]}


%%% ============================================================================================================= %%%
% Gray's New Commands
\newcommand\cdcout[1]{} % reduce it to 6 pages
%\newcommand\cdcout[1]{#1} % restores the CDC 7 page version


%%% ============================================================================================================= %%%
% Hong's Abbreviations for Left and Right Delimiters
\newcommand{\Lb}{\left [}          % Left bracket
\newcommand{\Rb}{\right ]}         % Right bracket
\newcommand{\LB}{\left \{}         % Left brace
\newcommand{\RB}{\right \}}        % Right brace
\newcommand{\Ld}{\left .}          % Left default
\newcommand{\Rd}{\right .}         % Right default
\newcommand{\Lp}{\left (}          % Left parenthese
\newcommand{\Rp}{\right )}         % Right parenthese
\newcommand{\Lv}{\left |}          % Left vertical line
\newcommand{\Rv}{\right |}         % Right vertical line
\newcommand{\LV}{\left \|}         % Left vertical double line
\newcommand{\RV}{\right \|}        % Right vertical double line


%%% ============================================================================================================= %%%
% Hong's New Commands and Declarations for Linear Algebra and Matrix Analysis
\DeclareMathOperator{\trace}{tr}   % Define a matrix' trace operator, sometimes
\DeclareMathOperator{\diag}{diag}  % Define the diagonal matrix operator
\DeclareMathOperator{\rank}{rank}  % Define the rank operator of a matrix
\DeclareMathOperator{\vect}{vec}   % Define the function to reshape a matrix into a vector column-wise.
% \DeclareMathOperator{\col}{col}    % Define the column operator, e.g., $cy = \col \LB y_i, \dots, y_n \RB$
\newcommand{\T}{^\mathrm{T}}       % Transposition operator of the matrix or vector
\newcommand{\h}{^\mathrm{H}}       % Hermitian Transposition of a matrix or vector


%%% ============================================================================================================= %%%
% Hong's New Commands and Declarations for Random Variables, Probability and Stochastic Process
\newcommand{\rv}[1]{\boldsymbol{#1}} % Use italic boldface to indicate the Random Variables
\DeclareMathOperator{\Var}{Var}      % Operator for variance of a RV
\DeclareMathOperator{\Cov}{Cov}      % Operator for covariance of a RV
\DeclareMathOperator*{\Lim}{l.i.m.}  % Operator for convergence in the MS sense

%%% ============================================================================================================= %%%
% Hong's New Commands and Declarations for Mathematical Analysis and Other Areas
\DeclareMathOperator{\sgn}{sgn}      % Operator for sign function
\DeclareMathOperator*{\argmax}{\arg\max}
%\DeclareMathOperator{\dist}{dist}

%%% ============================================================================================================= %%%
%% New Commands for Sets or Spaces
\newcommand{\N}{\mathbb N} % for the set of natural
\newcommand{\Q}{\mathbb Q} % for the set of rational
\newcommand{\R}{\mathbb R} % for the set of real
\newcommand{\Z}{\mathbb Z} % for the set of integer

%%% ============================================================================================================= %%%
% International Typesetting Standards
\newcommand{\me}{\mathrm{e}} % for math e
\newcommand{\mi}{\mathrm{i}} % for math i
\newcommand{\mj}{\mathrm{j}} % added by Hong for engineering j
\newcommand{\dif}{\,\mathrm{d}}% for differential

%%% ============================================================================================================= %%%
%% Hong's abbreviations of sets or spaces for this paper
\newcommand{\MRn}{\mathbb M(\R^{n})}
\newcommand{\Hn}{\mathbb H^n}
\newcommand{\Hnp}{\mathbb H^{n+}}
\newcommand{\BHn}{\mathbb B(\Hn)}

%%% ============================================================================================================= %%%
%% Hong's other settings just for this paper
%% \renewcommand{\labelenumi}{\textup{(\alph{enumi})}}
%% \setlength{\topsep}{-1pt}
%%\setlength{\itemsep}{-5pt}

%%% ============================================================================================================= %%%
%% Hong's other commands
\newcommand{\matlab}{\textsc{MATLAB}\textsuperscript{\textregistered}} % for the mathworks MATLAB registered product
\newcommand{\simulink}{Simulink\textsuperscript{\textregistered}} % for the mathworks Simulink registered product

%%% ============================================================================================================= %%%
%% Hong's Capital/Small Roman Numbers
\newcommand{\RomanNumber}[1]{\uppercase\expandafter{\romannumeral #1}}
\newcommand{\romannumber}[1]{\lowercase\expandafter{\romannumeral #1}}

%%% ============================================================================================================= %%%
%% Hong's Fonts definitions

\DeclareMathAlphabet{\mathpzc}{OT1}{pzc}{m}{it}

%% Table Raising box
\newcommand{\rb}[1]{\raisebox{1.5ex}[0pt]{#1}}

%%% ============================================================================================================= %%%
%% Dr. Gonzalez's Group definitions
\def\1{\rv 1} %Indicator
\def\I{{\mathcal I}}    %Index set
\def\E{{\mathbb E}}    %PDP state space
\def\bE{{\mathcal E}}    %Borel set
\def\x{{\rv\chi}}    %hybrid PDP state
\def\B{{\mathcal B}}    %Index set


%%% ============================================================================================================= %%%
% Luis' Changing Margins


 % Horizontal Spacing
% \setlength{\oddsidemargin}{-0.35in}
 %\setlength{\evensidemargin}{0.0in}
 %\setlength{\textwidth}{8.5in}
%
%Vertical Spacing
 %\renewcommand{\baselinestretch}{1.1}
 %\setlength{\textheight}{9.25in}
%\setlength{\headsep}{0.5in}



%%% ============================================================================================================= %%%
% Luis' Abbreviations

\def\allseriesPX{\mbox{$\re\langle\langle 
\mathfrak{P}X\rangle\rangle$} }
\def\allseriesPXell{\mbox{$\re^{\ell}\langle\langle 
\mathfrak{P}X\rangle\rangle$} }
\def\allseriesPXelln{\mbox{$\re^{\ell\times n}\langle\langle 
\mathfrak{P}X\rangle\rangle$} }
\def\allseriesPXnn{\mbox{$\re^{n\times n}\langle\langle 
\mathfrak{P}X\rangle\rangle$} }
\def\allseriesTX{\mbox{$\re\langle\langle 
\mathfrak{T}X\rangle\rangle$} }
\def\allseriesTXell{\mbox{$\re^{\ell}\langle\langle 
\mathfrak{T}X\rangle\rangle$} }
\def\allseriesTXelln{\mbox{$\re^{\ell\times n}\langle\langle 
\mathfrak{T}X\rangle\rangle$} }
\def\allpolyTXelln{\mbox{$\re^{\ell\times n}\langle 
\mathfrak{T}X\rangle$} }
\def\allseriesTXnn{\mbox{$\re^{n\times n}\langle\langle 
\mathfrak{T}X\rangle\rangle$} }
\def\allseriesTDXell{\mbox{$\re^{\ell}\langle\langle 
\mathfrak{TD}X\rangle\rangle$} }
\def\allseriesTDXelln{\mbox{$\re^{\ell\times n}\langle\langle 
\mathfrak{TD}X\rangle\rangle$} }
\def\allpolyTDX{\mbox{$\re\langle 
\mathfrak{TD}X\rangle$} }
\def\allpolyTDXelln{\mbox{$\re^{\ell\times n}\langle 
\mathfrak{TD}X\rangle$} }
\def\allpolyTXnn{\mbox{$\re^{n\times n}\langle 
\mathfrak{T}X\rangle$} }
\def\allpolyTX{\mbox{$\re\langle \mathfrak{T}X\rangle$} }
\def\allseriesY{\mbox{$\re\langle\langle Y \rangle\rangle$}}
\def\allseriesZ{\mbox{$\re\langle\langle Z \rangle\rangle$}}
\def\allseriestZ{\mbox{$\re\langle\langle \tilde{Z}\rangle\rangle$}}
\def\allseriesW{\mbox{$\re\langle\langle W\rangle\rangle$}}
\def\allseriesA{\mbox{$\re\langle\langle A\rangle\rangle$}}
\def\allseriesA'{\mbox{$\re\langle\langle A'\rangle\rangle$}}
\def\allseriesZj{\mbox{$\re\langle\langle Z_j\rangle\rangle$}}
\def\allseriesZbb{\mbox{$\re\langle\langle Z\cup{\bar{\bar{Z}}}\rangle\rangle$}}
\def\allseriesZZbZbb{\mbox{$\re\langle\langle Z\cup\bar{Z}\cup{\bar{\bar{Z}}}\rangle\rangle$}}
\def\allseriesXxY{\mbox{$\re\langle\langle X\otimes Y\rangle\rangle$}}
\def\allseriesXxX{\mbox{$\re\langle\langle X\otimes X \rangle\rangle$}}
\def\allseriesXXell{\mbox{$\re^{\ell}\langle\langle X\overline{X}\rangle\rangle$}}
\def\allseriesXXm{\mbox{$\re^{m}\langle\langle X\overline{X}\rangle\rangle$}}
\def\allseriesXX{\mbox{$\re\langle\langle X\overline{X}\rangle\rangle$}}
\def\allseriesXY{\mbox{$\re^{\ell}\langle\langle XY\rangle\rangle$}}
\def\allserieXY{\mbox{$\re\langle\langle XY\rangle\rangle$}}
\def\allseriesXYm{\mbox{$\re^{m}\langle\langle XY\rangle\rangle$}}
\def\allseriesXYntimesn{\mbox{$\re^{n\times n}\langle\langle XY\rangle\rangle$}}
\def\allseriesformalX{\mbox{$\re\langle\langle X_0\rangle\rangle$}}
\def\allseriesformalXY{\mbox{$\re\langle\langle X_0Y_0\rangle\rangle$}}
\def\allseriesformalXYell{\mbox{$\re^{\ell}\langle\langle X_0Y_0\rangle\rangle$}}
\def\allseriesformalXYm{\mbox{$\re^m\langle\langle X_0Y_0\rangle\rangle$}}
\def\allseriesformalXYmm{\mbox{$\re^{2m}\langle\langle X_0Y_0\rangle\rangle$}}
\def\allseriesXYmm{\mbox{$\re^{2m}\langle\langle XY\rangle\rangle$}}
\def\allseriesformalXYmLC{\mbox{$\re^m_{LC}\langle\langle X_0Y_0\rangle\rangle$}}
\def\allseriesformalXYellLC{\mbox{$\re^\ell_{LC}\langle\langle X_0Y_0\rangle\rangle$}}
\def\allseriesXYellLC{\mbox{$\re^{\ell}_{LC}\langle\langle XY \rangle\rangle$}}
\def\characserie#1{{\rv{#1}}} % Symbol for characteristic series
\def\Zint{{\mathbb Z}} % integer numbers (AMS symbol)
\def\allseriesY{\mbox{$\re\langle\langle Y\rangle\rangle$}}
\def\allseriesZ{\mbox{$\re\langle\langle Z\rangle\rangle$}}
\def\allseriesZbb{\mbox{$\re\langle\langle Z\cup{\bar{\bar{Z}}}\rangle\rangle$}}
\def\allseriesZZbZbb{\mbox{$\re\langle\langle Z\cup\bar{Z}\cup{\bar{\bar{Z}}}\rangle\rangle $}}
\def\allseriesfanX{\mbox{$\re^{\ell}\langle\langle \mathds{X}\rangle\rangle$}}
\def\allpolyZ{\mbox{$\re\langle Z\rangle$}}
\def\allpolyXY{\mbox{$\re\langle XY\rangle$}}
\def\allpolyZbb{\mbox{$\re\langle Z\cup{\bar{\bar{Z}}}\rangle$}}
\def\allpolyZZbZbb{\mbox{$\re\langle Z\cup\bar{Z}\cup{\bar{\bar{Z}}}\rangle $}}
\def\allseriesrat{\mbox{$\re^{rat}\langle\langle X\rangle $}}
\def\allpolyell{\mbox{$\re^{\ell}\langle X\rangle$}}
\def\allseriesxy{\mbox{$\re\langle\langle x,y\rangle\rangle $}}
\def\allliepolyX{\mbox{$\mathcal{L}(X)$}}
\def\allliepolyXY{\mbox{$\mathcal{L}(XY)$}}
\def\allpolycommutative{\mbox{$\re\left[ X \right]$}}
\def\allseriescommutativeXO{\mbox{$\re\left[[ X_0 ]\right]$}}
\def\allseriesgevreyr{\mbox{$\re_{G(r)}\langle\langle X \rangle\rangle$}}
\def\allseriesgevreyp{\mbox{$\re_{G(1/p)}\langle\langle X \rangle\rangle$}}
\def\allseriesgevreypr{\mbox{$\re_{G(1/p')}\langle\langle X \rangle\rangle$}}
\def\characserie#1{{\bf{#1}^*}} % Symbol for characteristic series
\def\Lspace{{ L}^2(\Omega,{\mathcal F}_0,P)}
\def\Lsquarespace{{ L}^2(\Omega,{\mathcal F},P)}
\def\Lsquare{{ L}^2(P)}
\def\L1spaceprodu{{ L}_1(\Omega\times [0,T],{\mathcal P},P\otimes \lambda)}
\def\Lspaceprodu{{ L}_2(\Omega\times [0,T],{\mathcal P},P\otimes \lambda)}
\def\Lpspaceprodu{{ L}_p(\Omega\times [0,T],{\mathcal P},P\otimes \lambda)}
\def\Lspaceprodum{{ L}_2^m(\Omega\times [0,T],{\mathcal P},P\otimes \lambda)}
\def\Lsquareprodu{{ L}_2(P\otimes \lambda)}
\def\Hsquare{{\mathcal H}^2}
\def\Hspace0{{\mathcal H}^2_0}
\def\Jsquare{{\mathcal J}^2}
\def\val{\mathfrak Val}
\def\calt{{\mathcal T}}
\def\calf{{\mathcal F}}
\def\calb{{\mathcal B}}
\def\calp{{\mathcal P}}
\def\E{\textbf{\textup  E}}
\def\filt{\textbf{\textup  F}}
\def\I{\textbf{\textup  I}}
\def\SS{\textbf{\textup  S}}
\def\qv#1{\left\langle#1\right\rangle}
\def\der#1#2#3{\frac{\partial#1}{\partial#2}{#3}}
\def\dersc#1#2#3{\frac{\partial^2#1}{\partial{#2}^2}{#3}}
\def\derscp#1#2#3#4{{\frac{\partial^2#1}{{\partial #2}{\partial #3}}}#4}
\def\intli{ \int\limits} 
\def\ints{\mathcal{S}  \hspace{-0.37cm}\int}
%\DeclareMathOperator{\ints}{\mathcal{S}  \hspace{-0.37cm}\int\limits}
\DeclareMathOperator{\intss}{\mathcal{^{_{_S}}} \hspace*{-0.31cm}\int\limits}
%\def\ints{\mbox{$ \mathcal{S} \!\!\!\!\!\! \hspace*{-0.02cm}\int\limits$}} 
\def\smallints{{^{_{_{\mathcal{S}}}}} \!\!\!\!\! \hspace*{-0.015cm}\int\limits}  % \ints for no \displaystyle
\def\inti{ \mathcal{{\rm I}} \hspace{-.34cm} \int\limits}
\def\normlr#1{\left\Vert#1\right\Vert}
\def\qlim{{\rm q}\!\!\! \lim}
\newcommand{\markblue}[1]{\textcolor[rgb]{0,0,1}{#1}}
\def\charseries{{\rm char}}

\def\shuffleNC{{ \prec \hspace*{-0.07cm}\succ}}
\newcommand{\arb}[1]{\begin{matrix}\includegraphics[height=7mm]{a#1.eps}
\end{matrix}}
\newcommand{\arblarge}[1]{\begin{matrix}\includegraphics[height=9mm]{a#1.eps}
\end{matrix}}
\newcommand{\arbsize}[1]{\begin{matrix}\includegraphics[height=14mm]{a#1.eps}
\end{matrix}}                      
\newcommand{\arbsmall}[1]{\begin{matrix}\includegraphics[height=3mm]{a#1.eps}
\end{matrix}}                                            
                      
\newcount\colveccount
\newcommand*\colvec[1]{
        \global\colveccount#1
        \begin{pmatrix}
        \colvecnext
}
\def\colvecnext#1{
        #1
        \global\advance\colveccount-1
        \ifnum\colveccount>0
                \\
                \expandafter\colvecnext
        \else
                \end{pmatrix}
        \fi
}

%\renewcommand{\baselinestretch}{1.5}   % comment for 1.5 spacing between lines


%% Luis Enviroment Definitions
\def\begpro{\noindent{\em Proof}$\;\;$}
\def\endpro{\hfill\bull \vspace*{0.1in}}


%%%%%%%%%%%%%%%%%%%%%%%%%%%%%%%%%%%%%%%%%%%%%
% Define a Lemma and theorem with a citation without parenthesis
% \newcounter{countlemm}
% \setcounter{countlemm}{0}
% \newenvironment{Lemma}[1]
% {\par \stepcounter{countlemm} {\itshape Lemma \arabic{lemma}} #1: \begin{itshape}}
% {\end{itshape}}
% \newcounter{countth}
% \setcounter{countth}{0}
% \newenvironment{Theorem}[1]
% {\par \stepcounter{countth} {\itshape Theorem \arabic{countth}} #1: \begin{itshape}}
% {\end{itshape}}
% \def\begle{\begin{Lemma}}
% \def\endle{\end{Lemma}}
% \def\begth{\begin{Theorem}}
% \def\endth{\end{Theorem}}
%%%%%%%%%%%%%%%%%%%%%%%%%%%%%%%%%%%%%%%%%%%%%

\renewcommand{\baselinestretch}{.96}
